\chapter{Einführung}

Ein wichtiger Bestandteil eines sozialen Wandlungsprozesses ist der Prozess der Digitalisierung im Gesundheitswesen, denn die generierte Datenmenge ist heutzutage nicht im Papierformat zu bewältigen. Dieser Prozess bringt mit sich Herausforderungen, die zu bewältigen sind, sodass die Nutzung moderner \ac{it}-Technologien und Standards im Gesundheitswesen, in Bezug auf eine Verbesserung der Versorgung und Forschung in dem Gesundheitssystem ermöglicht wird. Noch dazu übt die Digitalisierung auch einen Einfluss auf die Entwicklung der Interaktion zwischen unterschiedlichen an der gesundheitlichen Versorgung beteiligten Instanzen aus. 

Eine der zentralen Herausforderungen in dem Prozess der Digitalisierung im Gesundheitswesen ist, zusammen mit dem gewaltigen generierte Datenvolumen, die mangelnde Interoperabilität vieler Systeme, denn viele Unternehmen haben eigene Lösungen für einzelne Komponenten herstellt, sodass die Interaktion von Systemen in einem Standort oder die Kommunikation zwischen verschiedenen Standorten in vielen Fällen unpraktikable ist. Andere Problematik der mangelnden Interoperabilität ist die dazu mangelnde Nutzbarkeit der Daten für die Versorgung und diverse Forschungsprojekte, die auch die Krankenversorgung in nähre Zukunft fördern könnten. Aus diesem Grund soll innerhalb dieses Dokument ein erstes Versuch einer Zuordnung-Prozesses von \ac{copra}-Konfigurationsvariablen aus der Routineversorgung an der Universitätsmedizin Mainz mit \ac{loinc}-Codes dokumentiert werden. 

Um diese Ziel zu erreichen, wurden \ac{loinc}-Codes zu den \ac{copra}-Konfigurations- variablen benutzt. %Dazu wurden etablierte Codesysteme wie \ac{loinc} und \ac{snomedct} zusammen mit \ac{it}-Werkzeugen wie \ac{sql}, \ac{regex}, \ac{etl}-Prozessen, etc. angewendet.

%Dieses Dokument stellt den Prozess des Mappings der \ac{copra}-Konfigurationsvariablen mit \ac{loinc}-Codes dar.

%Im \ac{diz} werden Daten aus verschiedenen Fachabteilungen und Systemen zusammengeführt. Ein zentrales Puzzleteil für die Zwischenspeicherung der Information dieser Systemen ist das \acf{csdwh}. In dieser \ac{db} werden alle relevanten klinischen Systeme abgebildet. Diese Daten werden im Rahmen des Datenschutz sowie der Datenqualität aufbereitet und anschließend an weitere Komponenten des \ac{diz} übertragen.